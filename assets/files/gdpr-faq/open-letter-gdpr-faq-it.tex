\documentclass{scrlttr2}
\KOMAoptions{
  foldmarks=true,
  fromalign=left,
  % fromrule=afteraddress,
  fromemail,
  % fromphone,
  % parskip=full
  parskip=half,
  firsthead=false,
  enlargefirstpage=true,
  % addrfield=topaligned,
  addrfield=off,
  DIV=16,
}
\usepackage[UKenglish]{babel}
\usepackage[utf8]{inputenc}
\usepackage[T1]{fontenc}
\usepackage{graphicx}
\usepackage{marvosym}
\usepackage[usenames,dvipsnames]{xcolor}
\usepackage{pdfpages}
\usepackage{hyperref}
\hypersetup{
%   colorlinks  = true,
%   urlcolor  = darkblue,
%   pdftitle  = {},
%   pdfsubject  = {},
%   pdfauthor = {Robert Riemann},
%   pdfkeywords = {},
  pdfcreator  = {pdftex},
  pdfproducer = {pdftex}
}
\usepackage{microtype}
%\usepackage{garamondx}
\usepackage{charter}

\usepackage{lastpage}
\usepackage{scrpage2}
% ===========================
%    HEAD- AND FOOTLINES
% ===========================
\pagestyle{scrheadings}
\clearscrheadfoot
\ifoot{\usekomavar{subject}}
\ofoot{\thepage\ of \pageref{LastPage}}
\ihead{}
\ohead{}

\makeatletter
% \@addtoplength{refvpos}{-6cm}
\@setplength{refvpos}{\useplength{toaddrvpos}}
\makeatother

\let\tightlist\relax

\begin{document}

\setkomavar{place}{Bruxelles}

\setkomavar{backaddressseparator}{\ \textperiodcentered\ }
\setkomavar{signature}{\\[-2\baselineskip]\parbox[t]{\linewidth}{\raggedright Robert Riemann, Brussels\\Xavier Lavayssière, Paris\\Franz Ritschel, Köln}}
\setkomavar{date}{6 giugno 2018}
\setkomavar{subject}{Lettera aperta: appello per uno spazio collaborativo di FAQ sulla
protezione dei dati}

% \let\oldhref\href\renewcommand{\href}[2]{\oldhref{#1}{#2}\footnote{\url{#1}}}

\begin{letter}{}
\opening{Cari professionisti della protezione dei dati e dell'IT,}

Il nuovo Regolamento Generale sulla Protezione dei Dati (in breve RGPD)
si applica a decorrere dal 25 maggio 2018. È costituito da 99 articoli e
173 considerando che occupano 88 pagine nella
\href{http://eur-lex.europa.eu/legal-content/EN/TXT/?uri=uriserv:OJ.L_.2016.119.01.0001.01.ENG}{pubblicazione
ufficiale}. A differenza di un documento di normalizzazione tecnica,
molti di questi articoli devono essere prima interpretati alla luce
della giurisprudenza già consolidata e dei pareri precedentemente
emanati dalle autorità di protezione dei dati. Di conseguenza, persino
questioni di conformità per applicazioni relativamente semplici come una
mailing list non possono essere risolti senza uno studio approfondito di
diversi documenti giuridici. Concetti complessi come la \emph{privacy by
design} e la \emph{pseudonimizzazione} sono fonte di molte domande a cui
si deve ancora rispondere.

Allo stesso tempo, da molti anni nel settore dell'alta tecnologia si
lavora a soluzioni agili che permettano la raccolta e il trattamento dei
dati personali. Grazie a \emph{Google Sheets}, \emph{Doodle},
\emph{Mailchimp}, o \emph{Wordpress}, anche i non esperti, oggigiorno,
possono diventare responsabili del trattamento di dati personali in
pochi clic o passaggi. Lo sviluppo di protocolli peer-to-peer per
database modulabili, come \emph{Bitcoin},
\emph{\href{https://datproject.org/}{Dat}} o
\emph{\href{https://ipfs.io/}{IPFS}}, è in grado di ridurre
ulteriormente le difficoltà iniziali per diventare responsabile del
trattamento di dati, fino ad uno stadio di inconsapevolezza del
responsabile.

Per consentire una rapida adozione degli obblighi di protezione dei dati
e, di conseguenza, un aumento generale della qualità dei dati, è
necessaria una formazione per i responsabili del trattamento dei dati e
gli incaricati del trattamento e deve essere accessibile non solo a
coloro che possono permettersi di dedicarvi le risorse. Per questo
motivo, chiediamo la creazione di una banca dati di conoscenze
collaborativa su internet sotto la licenza
\href{https://creativecommons.org/}{Creative Commons} per garantire la
sua ampia e continua disponibilità.

Finora, i consigli pratici liberamente accessibili sono spesso, se non
per lo più, offerti da parti interessate che possono avere interessi
commerciali in conflitto. Fornitori di servizi online, studi legali e
istituti di formazione possono orientare la consulenza verso i propri
servizi. Le licenze restrittive possono impedire che un buon consiglio
venga condiviso liberamente. I consigli errati o datati potrebbero non
essere aggiornati. Quest'ultimo punto è particolarmente importante in
quanto la conformità al RGPD è un obiettivo mobile. Nuove sentenze o
avanzamenti tecnici in materia di protezione della vita
privata\footnote{Il RGPD richiede all'art.~25, dedicato alla protezione
  dei dati fin dalla progettazione e protezione per impostazione
  predefinita, che i titolari del trattamento dei dati debbano tener
  conto, tra le altre cose, dello stato dell'arte al momento di
  determinare i mezzi del trattamento e all'atto del trattamento stesso.}
richiedono aggiornamenti continui.

Poiché la protezione dei dati è un settore interdisciplinare, la banca
dati di conoscenza dovrebbe essere redatta congiuntamente da giuristi e
ingegneri informatici e soddisfare le esigenze di entrambe le comunità.
La piattaforma \emph{Stack Exchange} fornisce alle comunità una
soluzione software per domande frequenti (FAQ) collaborative. La
piattaforma è ben nota alla maggior parte degli ingegneri informatici
via \href{https://stackoverflow.com}{stackoverflow.com} e ha avviato più
recentemente
\href{https://law.stackexchange.com}{law.stackexchange.com}\footnote{\href{https://law.stackexchange.com}{law.stackexchange.com}
  include già domande sul
  \href{https://law.stackexchange.com/questions/tagged/gdpr}{RGPD} e la
  \href{https://law.stackexchange.com/questions/tagged/gdpr+data-protection}{protezione
  dei dati}. Tuttavia, riteniamo che la protezione dei dati personali
  meriti una sua propria piattaforma che comprenda anche altre
  discipline come l'ingegneria informatica o l'etica.}. La
collaborazione è organizzata come segue:

\begin{itemize}
\tightlist
\item
  Le domande, le risposte e i loro metadati sono pubblicati su internet
  sotto una licenza gratuita
  (\href{https://creativecommons.org/licenses/by-sa/3.0/}{cc by-sa}) e
  sono scaricabili e leggibili da dispositivo elettronico.
\item
  Chiunque può chiedere o rispondere a una domanda.
\item
  Le risposte migliori vengono votate.
\item
  Gli utenti guadagnano in reputazione per ogni voto che ricevono.
\item
  Gli utenti sbloccano dei vantaggi attraverso la reputazione
  guadagnata, così come la possibilità di commentare o votare.
\item
  I moderatori sono eletti tra gli utenti e i principali utenti hanno
  accesso a strumenti speciali per aiutare a moderare.
\end{itemize}

Per fornire un'alta qualità globale delle risposte, i riferimenti alle
fonti principali dovrebbero essere utilizzati laddove vi siano dei punti
soggetti ad opinioni divergenti. Questa regola è anche utilizzata da
Wikipedia e può essere applicata dagli moderatori e dagli utenti
principali.

I firmatari supportano la creazione di tale banca dati collaborativa di
conoscenze sulla protezione dei dati collaborativo sotto forma di
domande frequenti.

\textbf{Autori e firmatari iniziali:}

\begin{itemize}
\tightlist
\item
  Robert Riemann, Brüssel
\item
  Xavier Lavayssière, Paris
\item
  Franz Ritschel, Köln
\end{itemize}

\textbf{Contatto:}

Se si desidera ricevere aggiornamenti o se si hanno domande, si prega di
inviare la richiesta a
\href{mailto:gdpr-faq@riemann.cc}{\nolinkurl{gdpr-faq@riemann.cc}}. Se
si desidera diventare firmatario, si prega di inviare una mail a
\href{mailto:gdpr-faq-sign@riemann.cc}{\nolinkurl{gdpr-faq-sign@riemann.cc}}.
Le richieste in lingua francese sono da indirizzare a
\href{mailto:gdpr-faq@lesbricodeurs.fr}{\nolinkurl{gdpr-faq@lesbricodeurs.fr}}
e per la firma a
\href{mailto:gdpr-faq-signer@lesbricodeurs.fr}{\nolinkurl{gdpr-faq-signer@lesbricodeurs.fr}}.

\newpage

\textbf{Elenco dei destinatari:}

\begin{itemize}
\tightlist
\item
  \href{https://edps.europa.eu/data-protection/ipen-internet-privacy-engineering-network_en}{Internet
  Privacy Engineering Network} (IPEN), un'iniziativa del garante europeo
  della protezione dei dati
\item
  Stack Overflow, la società dietro il famoso database di conoscenze
  \href{https://stackoverflow.com}{stackoverflow.com} per i
  programmatori
\item
  \href{https://edri.org/}{European Digital Rights} (EDRi),
  un'associazione di organizzazioni a difesa dei diritti civili e umani
  di tutta Europa
\item
  i partecipanti all'edizione 2018
  dell'\href{http://privacyforum.eu/}{Annual Privacy Forum} (APF)
\item
  il comitato organizzativo della conferenza internazionale
  \href{http://www.cpdpconferences.org/}{Computer, Privacy and Data
  Protection} (CPDP)
\item
  una lista di singoli destinatari
\end{itemize}

\end{letter}

\end{document}