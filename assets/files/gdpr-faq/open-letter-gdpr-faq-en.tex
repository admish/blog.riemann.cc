\documentclass{scrlttr2}
\KOMAoptions{
  foldmarks=true,
  fromalign=left,
  % fromrule=afteraddress,
  fromemail,
  % fromphone,
  % parskip=full
  parskip=half,
  firsthead=false,
  enlargefirstpage=true,
  % addrfield=topaligned,
  addrfield=off,
  DIV=16,
}
\usepackage[UKenglish]{babel}
\usepackage[utf8]{inputenc}
\usepackage[T1]{fontenc}
\usepackage{graphicx}
\usepackage{marvosym}
\usepackage[usenames,dvipsnames]{xcolor}
\usepackage{pdfpages}
\usepackage{hyperref}
\hypersetup{
%   colorlinks  = true,
%   urlcolor  = darkblue,
%   pdftitle  = {},
%   pdfsubject  = {},
%   pdfauthor = {Robert Riemann},
%   pdfkeywords = {},
  pdfcreator  = {pdftex},
  pdfproducer = {pdftex}
}
\usepackage{microtype}
%\usepackage{garamondx}
\usepackage{charter}

\usepackage{lastpage}
\usepackage{scrpage2}
% ===========================
%    HEAD- AND FOOTLINES
% ===========================
\pagestyle{scrheadings}
\clearscrheadfoot
\ifoot{\usekomavar{subject}}
\ofoot{\thepage\ of \pageref{LastPage}}
\ihead{}
\ohead{}

\makeatletter
% \@addtoplength{refvpos}{-6cm}
\@setplength{refvpos}{\useplength{toaddrvpos}}
\makeatother

\let\tightlist\relax

\begin{document}

\setkomavar{place}{Brussels}

\setkomavar{backaddressseparator}{\ \textperiodcentered\ }
% \setkomavar{signature}{\\[-2\baselineskip]\parbox[t]{\linewidth}{\raggedright Robert Riemann, Brussels\\Xavier Lavayssière, Paris\\Franz Ritschel, Köln}}
\setkomavar{date}{6th June 2018}
\setkomavar{subject}{Open Letter: Call for a collaborative data protection FAQ}

% \let\oldhref\href\renewcommand{\href}[2]{\oldhref{#1}{#2}\footnote{\url{#1}}}

\begin{letter}{}
\opening{Dear Data Protection and IT Professionals,}


The EU's new law \emph{General Data Protection Regulation} (GDPR for
short) applies from 25 May 2018 onwards. It consists of 99~articles and
173~recitals that fill together 88~pages
\href{http://eur-lex.europa.eu/legal-content/EN/TXT/?uri=uriserv:OJ.L_.2016.119.01.0001.01.ENG}{in
the official publication}. Different than a technical standardisation
document, many of those articles must first be interpreted under consideration of
case law from past judgements and published opinions of data protection
authorities. As a result, even compliance questions for relatively
simple applications such as a mailing list cannot be answered without
profound study of many legal documents. Complex concepts such as
\emph{privacy by design} and \emph{pseudonymisation} are the source for
many questions yet to be answered.

At the same time, the tech industry has worked for many years on
solutions to setup fairly easy personal data processing applications.
Thanks to e.g. \emph{Google Sheets}, \emph{Doodle}, \emph{Mailchimp}, or
\emph{Wordpress}, even non-experts can nowadays become data controllers
with only few clicks or swipes. The development of peer-to-peer
protocols for distributed databases, e.g. \emph{Bitcoin},
\emph{\href{https://datproject.org/}{Dat}}, or
\emph{\href{https://ipfs.io/}{IPFS}}, has the potential to further lower
the initial hurdle to become a data controller---up to the point of
unconsciousness of the controller.

To allow for a rapid adoption of data protection obligations, and in
turn an overall increase of data hygiene, training for data controllers
and processors is needed and must be accessible not only for those who
can afford to dedicate resources, but at best to all data controllers
and processors. For this reason, we call for the foundation of a
collaborative Internet knowledge database under a free
\href{https://creativecommons.org/}{creative commons} license to ensure
its broad and continuous availability.

So far, freely accessible practical advice is often, if not mostly,
offered by stakeholders that may have conflicting business interests.
Online service providers, law firms and training institutes may gear
advice towards their own services. Restrictive licenses may prevent good
advice from being freely shared. Erroneous or out-dated advice may not
be updated. Especially, the latter is important as GDPR compliance is a
moving target. New judgements or advances in state-of-the-art privacy
engineering\footnote{The GDPR mandates in Art.~25 on data protection by
  design and by default controllers of data processing to take into
  account among others the state of the art when defining means for data
  processing and during the data processing itself.} require continuous
updates.

As data protection is an interdisciplinary field, the knowledge database
should be co-authored jointly by legal experts and computer engineers
and must accommodate the needs of both communities. The platform \emph{Stack
Exchange} provides communities with a software solution for collaborative
freqently asked questions (FAQ). The platform is well-known to most
computer engineers for offering
\href{https://stackoverflow.com}{stackoverflow.com} and started more
recently
\href{https://law.stackexchange.com}{law.stackexchange.com}\footnote{\href{https://law.stackexchange.com}{law.stackexchange.com}
  covers already questions on
  \href{https://law.stackexchange.com/questions/tagged/gdpr}{GDPR} and
  \href{https://law.stackexchange.com/questions/tagged/gdpr+data-protection}{data
  protection}. However, we feel that data protection deserves its own
  platform that encompasses also other disciplines such as computer
  engineering or ethics.}. The collaboration is organised as follows:

\begin{itemize}
\tightlist
\item
  Questions, answers and meta-data are published in the Internet under a
  free license
  (\href{https://creativecommons.org/licenses/by-sa/3.0/}{cc by-sa}) and
  are available for download in machine-readable form.
\item
  Anybody can ask or answer a question.
\item
  The best answers are voted to the top.
\item
  Users earn reputation points for every vote they receive.
\item
  Users unlock privileges as they earn reputation, like the ability to
  comment or vote.
\item
  Moderators are elected among users, and top users have access to
  special tools to help moderate.
\end{itemize}

To provide for an overall high quality of answers, references to primary
sources shall be used where opinions are inevitable. This rule is also
employed by Wikipedia and can be enforced by both moderators and top
users.

The signatories support the foundation of such a collaborative data
protection knowledge database in form of frequently asked questions.

\textbf{Authors and Initial Signatories:}

\begin{itemize}
\tightlist
\item
  Robert Riemann, Brussels
\item
  Xavier Lavayssière, Paris
\item
  Franz Ritschel, Köln
\end{itemize}


\textbf{Contact:}

If you want to receive updates or if you have questions, please send
your request to
\href{mailto:gdpr-faq@riemann.cc}{\nolinkurl{gdpr-faq@riemann.cc}}. If
you want to become a signatory, send a mail to
\href{mailto:gdpr-faq-sign@riemann.cc}{\nolinkurl{gdpr-faq-sign@riemann.cc}}.
Requests in French language are answered at
\href{mailto:gdpr-faq@lesbricodeurs.fr}{\nolinkurl{gdpr-faq@lesbricodeurs.fr}}
and for signing at
\href{mailto:gdpr-faq-signer@lesbricodeurs.fr}{\nolinkurl{gdpr-faq-signer@lesbricodeurs.fr}}.

\newpage

\textbf{List of Recipients:}

\begin{itemize}
\tightlist
\item
  the
  \emph{\href{https://edps.europa.eu/data-protection/ipen-internet-privacy-engineering-network_en}{Internet
  Privacy Engineering Network}} (IPEN for short), an initiative of the
  European Data Protection Supervisor
\item
  \emph{\href{https://stackoverflow.com/company}{Stack Overflow}}, the
  company behind the famous knowledge database
  \href{https://stackoverflow.com}{stackoverflow.com} for programmers
\item
  the \href{https://edri.org/}{European Digital Rights} (EDRi for
  short), an association of civil and human rights organisations from
  across Europe
\item
  the participants of the 2018 edition of the
  \href{http://privacyforum.eu/}{Annual Privacy Forum} (APF for short)
\item
  the organisation committee of the international conference
  \href{http://www.cpdpconferences.org/}{Computers, Privacy and Data
  Protection} (CPDP for short)
\item
  and non-disclosed individual recipients
\end{itemize}


\end{letter}

\end{document}
