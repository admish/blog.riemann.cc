\documentclass{scrlttr2}
\KOMAoptions{
  foldmarks=true,
  fromalign=left,
  % fromrule=afteraddress,
  fromemail,
  % fromphone,
  % parskip=full
  parskip=half,
  firsthead=false,
  enlargefirstpage=true,
  % addrfield=topaligned,
  addrfield=off,
  DIV=16,
}
\usepackage[ngerman]{babel}
\usepackage[utf8]{inputenc}
\usepackage[T1]{fontenc}
\usepackage{graphicx}
\usepackage{marvosym}
\usepackage[usenames,dvipsnames]{xcolor}
\usepackage{pdfpages}
\usepackage{hyperref}
\hypersetup{
%   colorlinks  = true,
%   urlcolor  = darkblue,
%   pdftitle  = {},
%   pdfsubject  = {},
%   pdfauthor = {Robert Riemann},
%   pdfkeywords = {},
  pdfcreator  = {pdftex},
  pdfproducer = {pdftex}
}
\usepackage{microtype}
%\usepackage{garamondx}
\usepackage{charter}

\usepackage{lastpage}
\usepackage{scrpage2}
% ===========================
%    HEAD- AND FOOTLINES
% ===========================
\pagestyle{scrheadings}
\clearscrheadfoot
\ifoot{\usekomavar{subject}}
\ofoot{\thepage\ of \pageref{LastPage}}
\ihead{}
\ohead{}

\makeatletter
% \@addtoplength{refvpos}{-6cm}
\@setplength{refvpos}{\useplength{toaddrvpos}}
\makeatother

\let\tightlist\relax

\begin{document}

\setkomavar{place}{Brüssel}

\setkomavar{backaddressseparator}{\ \textperiodcentered\ }
\setkomavar{signature}{\\[-2\baselineskip]\parbox[t]{\linewidth}{\raggedright Robert Riemann, Brussels\\Xavier Lavayssière, Paris\\Franz Ritschel, Köln}}
\setkomavar{date}{6th June 2018}
\setkomavar{subject}{Offener Brief: Aufruf zur Gründung einer kollaborativen FAQ für Datenschutz}

% \let\oldhref\href\renewcommand{\href}[2]{\oldhref{#1}{#2}\footnote{\url{#1}}}

\begin{letter}{}
\opening{Sehr geehrte Datenschutz- und IT-Expert\_innen,}

Die neue EU-Datenschutz-Grundverordnung (DSGVO) gilt ab dem 25. Mai
2018. Sie besteht aus 99~Artikeln und 173~Erwägungsgründen und umfasst
88~Seiten
\href{http://eur-lex.europa.eu/legal-content/EN/TXT/?uri=uriserv:OJ.L_.2016.119.01.0001.01.ENG}{in
der amtlichen Fassung}. Anders als technische Normen ist die DSGVO ein
Gesetz und wird von der Rechtsprechung und den Rechtsanwender, allen
voran den Datenschutzbehörden, durch Urteile bzw. Stellungnahmen
ausgelegt. Dadurch können auch Fragen zu simplen Anwendungen wie
Mailinglisten nicht ohne gründliches Studium vieler Rechtsdokumente
beantwortet werden. Komplexe Konzepte wie \emph{Privacy by Design} oder
\emph{Pseudonymisierung} sind erst recht Quelle vieler Fragen, die es zu
beantworten gilt.

Gleichzeitig arbeiten Technologiefirmen schon seit Jahren an Lösungen,
um die Verarbeitung von persönlichen Daten relativ einfach zu gestalten.
Dank \emph{Google Sheets}, \emph{Doodle}, \emph{Mailchimp} oder
\emph{Wordpress} können heutzutage auch Nicht-Experten mit wenigen
Klicks zu Verantwortlichen im Sinne der DSGVO zu werden.
Peer-to-Peer-Protokolle für verteilte Datenbanken, z.B. Bitcoin,
\emph{\href{https://datproject.org/}{Dat}} oder
\emph{\href{https://ipfs.io/}{IPFS}} könnten die Zugangsbarrieren weiter
abbauen---bis hin zur Unmerklichkeit der Verarbeitung seitens der
Verantwortlich\_innen.

Um aber Datenschutzverpflichtungen schnell und wirksam übernehmen zu
können und um damit insgesamt einen Beitrag zu einer höheren
Datensicherheit zu leisten, sind wohl oder übel Schulungen für
Verantwortliche und Auftragsverarbeiter nötig. Diese dürfen nicht nur
wenigen Profis zugänglich sein, sondern sollten allen Verantwortlichen
und Auftragsverarbeitern offen stehen. Aus diesem Grund rufen wir zur
Gründung einer englisch-sprachigen, kollaborativen
Internet-Wissensdatenbank auf, die unter
\href{https://creativecommons.org/}{freier Lizenz} betrieben werden soll
um eine hohe Reichweite zu ermöglichen.

Bislang wurden frei zugängliche praktische Ratschläge oft, wenn nicht
sogar überwiegend, von Dienstanbietern angeboten, die eigene, eventuell
entgegengesetzte, Geschäftsinteressen verfolgen mögen. So geben etwa
viele Online-Diensteanbieter, Anwaltskanzleien oder Ausbildungsinstitute
Ratschläge um auch eigene Diensteistungen zu bewerben. Restriktive
Lizenzen der Ratschläge verhindern, dass guter Rat kostenlos
weitergegeben werden kann. Fehlerhafte und veraltete Ratschläge können
zumeist nicht verbessert werden. Letzteres ist besonders wichtig, da die
Einhaltung der DSGVO ein bewegliches Ziel ist, denn durch neue Urteile
oder Fortschritte in der modernen Privatsphäre entwickelt sich der
Datenschutz ständig weiter\footnote{Die DSGVO verlangt von
  Verantwortlichen in Art.~25 über Datenschutz durch Technikgestaltung
  und durch datenschutzfreundliche Voreinstellungen den gegenwertigen
  Stand der Technik zu berücksichtigen, wenn Datenverarbeitung geplannt
  wird oder bereits statt findet.}. Ratschläge müssen deshalb
kontinuierlich angepasst werden können.

Da Datenschutz ein interdisziplinäres Feld ist, sollte die
Wissensdatenbank gemeinsam von Rechts- und IT-Experten angelegt werden
und muss daher die Bedüfnisse beider Gruppen beachten. Die Plattform
\emph{Stack Exchange} bietet für die Beantwortung häufig gestellter
Fragen (FAQ) eine passende Softwarelösung. Die Plattform ist den meisten
IT-Experten bereits von der Seite
\href{https://stackoverflow.com}{stackoverflow.com} vertraut und bietet
mit
\href{https://law.stackexchange.com}{law.stackexchange.com}\footnote{\href{https://law.stackexchange.com}{law.stackexchange.com}
  listet bereits Fragen zur
  \href{https://law.stackexchange.com/questions/tagged/gdpr}{DSGVO} und
  \href{https://law.stackexchange.com/questions/tagged/gdpr+data-protection}{Datenschutz}.
  Jedoch finden wir, dass Datenschutz eine eigene Plattform verdient um
  andere Fachbereiche wie Informatik und Ethik besser einzubeziehen.}
seit kurzem auch ein englisch-sprachiges Angebot für Rechts-Experten.
Die Zusammenarbeit ist sets wie folgt organisiert:

\begin{itemize}
\tightlist
\item
  Fragen, Antworten und Metadaten werden im Internet unter einer freien
  Lizenz (\href{https://creativecommons.org/licenses/by-sa/3.0/}{cc
  by-sa}) veröffentlicht und stehen in maschinenlesbarer Form zum
  Download bereit.
\item
  Jede Person kann eine Frage stellen oder beantworten.
\item
  Die besten Antworten werden an die Spitze gewählt.
\item
  Nutzer\_innen erhalten für jede abgegebene Stimme Reputationspunkte.
\item
  Benutzer entsperren Privilegien, wenn sie sich Reputation verdienen,
  und können dann zum Beispiel Inhalte kommentieren oder bewerten.
\item
  Moderatoren werden unter Benutzern ausgewählt, und Top-Benutzer
  bekommen die Möglichkeit bei der Moderation zu helfen.
\end{itemize}

Um eine insgesamt hohe Qualität der Antworten zu gewährleisten, soll auf
Primärquellen verwiesen werden, wenn Meinungen unvermeidbar sind. Diese
Regel wird unter anderem von Wikipedia angewendet und kann sowohl von
Moderatoren als auch von Top-Benutzern durchgesetzt werden.

Die Unterzeichner unterstützen die Gründung einer solchen kollaborativen
Wissensdatenbank zum Thema Datenschutz in Form von häufig gestellten
Fragen (FAQ).

\textbf{Autoren und Erstunterzeichner:}

\begin{itemize}
\tightlist
\item
  Robert Riemann, Brüssel
\item
  Xavier Lavayssière, Paris
\item
  Franz Ritschel, Köln
\end{itemize}

\textbf{Kontakt:}

Wenn Sie Updates erhalten möchten oder Fragen haben, senden Sie bitte
Ihre Anfrage an
\href{mailto:gdpr-faq@riemann.cc}{\nolinkurl{gdpr-faq@riemann.cc}}. Wenn
Sie Unterzeichner werden möchten, senden Sie eine E-Mail an
\href{mailto:gdpr-faq-sign@riemann.cc}{\nolinkurl{gdpr-faq-sign@riemann.cc}}.
Anfragen in Französischer Sprache werden von
\href{mailto:gdpr-faq@lesbricodeurs.fr}{\nolinkurl{gdpr-faq@lesbricodeurs.fr}}
beantwortet and von
\href{mailto:gdpr-faq-signer@lesbricodeurs.fr}{\nolinkurl{gdpr-faq-signer@lesbricodeurs.fr}}
um Unterzeicher zu werden.

\newpage

\textbf{Liste der Empfänger:}

\begin{itemize}
\tightlist
\item
  das \href{https://edps.europa.eu/data-protection/ipen-internet-privacy-engineering-network_en}{Internet Privacy Engineering Network} (kurz IPEN), eine Initiative des Europäischen Datenschutzbeauftragten EDPS
\item
  Stack Overflow, das Unternehmen hinter der bekannten Wissensdatenbank
  \href{https://stackoverflow.com}{stackoverflow.com} für IT-Experten
\item
  der Verband \href{https://edri.org/}{European Digital Rights} (kurz
  EDRi) von Bürger- und Menschenrechtsorganisationen aus ganz Europa
\item
  die Teilnehmer des \href{http://privacyforum.eu/}{Annual Privacy Forums} (kurz APF) 2018
\item
  das Organisationskomitee der internationalen Konferenz
  \href{http://www.cpdpconferences.org/}{Computer, Privacy and Data
  Protection} (kurz CPDP)
\item
  und weitere einzelne Personen
\end{itemize}

\end{letter}

\end{document}