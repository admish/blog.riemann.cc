\documentclass{scrlttr2}
\KOMAoptions{
  foldmarks=true,
  fromalign=left,
  % fromrule=afteraddress,
  fromemail,
  % fromphone,
  % parskip=full
  parskip=half,
  firsthead=false,
  enlargefirstpage=true,
  % addrfield=topaligned,
  addrfield=off,
  DIV=16,
}
\usepackage[UKenglish]{babel}
\usepackage[utf8]{inputenc}
\usepackage[T1]{fontenc}
\usepackage{graphicx}
\usepackage{marvosym}
\usepackage[usenames,dvipsnames]{xcolor}
\usepackage{pdfpages}
\usepackage{hyperref}
\hypersetup{
%   colorlinks  = true,
%   urlcolor  = darkblue,
%   pdftitle  = {},
%   pdfsubject  = {},
%   pdfauthor = {Robert Riemann},
%   pdfkeywords = {},
  pdfcreator  = {pdftex},
  pdfproducer = {pdftex}
}
\usepackage{microtype}
%\usepackage{garamondx}
\usepackage{charter}

\usepackage{lastpage}
\usepackage{scrpage2}
% ===========================
%    HEAD- AND FOOTLINES
% ===========================
\pagestyle{scrheadings}
\clearscrheadfoot
\ifoot{\usekomavar{subject}}
\ofoot{\thepage\ of \pageref{LastPage}}
\ihead{}
\ohead{}

\makeatletter
% \@addtoplength{refvpos}{-6cm}
\@setplength{refvpos}{\useplength{toaddrvpos}}
\makeatother

\let\tightlist\relax

\begin{document}

\setkomavar{place}{Bruxelles}

\setkomavar{backaddressseparator}{\ \textperiodcentered\ }
\setkomavar{signature}{\\[-2\baselineskip]\parbox[t]{\linewidth}{\raggedright Robert Riemann, Bruxelles\\Xavier Lavayssière, Paris\\Franz Ritschel, Cologne}}
\setkomavar{date}{6 Juin 2018}
\setkomavar{subject}{Lettre ouverte pour l'ouverture d'un espace RGPD}

% \let\oldhref\href\renewcommand{\href}[2]{\oldhref{#1}{#2}\footnote{\url{#1}}}

\begin{letter}{}
\opening{Chers professionnels de la protection des données et de
l'informatique}

Le \emph{Règlement Général de Protection des données}~(RGPD) s'applique
depuis le 25~mai 2018. Ce règlement européen consiste en 99~articles et
173~considérants qui atteignent 88~pages dans la
\href{http://eur-lex.europa.eu/legal-content/EN/TXT/?uri=uriserv:OJ.L_.2016.119.01.0001.01.ENG}{publication
officielle}. Contrairement à un document de standardisation technique,
les articles de ce règlement doivent être interprétés en tenant compte
de la jurisprudence et des opinions des spécialistes. En conséquence, la
mise en conformité de systèmes simples comme une liste de diffusion ne
peut pas être effectuée sans l'étude préalable de plusieurs documents
juridiques. Les concepts complexes comme la \emph{privacy by design} et
la \emph{pseudonymisation} soulèvent de nombreuses questions.

Pendant ce temps, les entreprises développent des solutions qui
permettent de recueillir et manipuler facilement des données
personnelles. Grâce à \emph{Google Sheets}, \emph{Doodle},
\emph{Mailchimp} ou \emph{Wordpress} des non spécialistes peuvent
devenir \emph{responsables de traitement}\footnote{Personne morale ou
  physique qui détermine les finalités et les moyens du traitement,
  c'est-à-dire toute opération, effectuée ou non à l'aide de procédés
  automatisés, et appliquées à des données à caractère personnel (
  Article 4 de la RGPD )} en quelques clics. Le développement de
protocoles de pair à pair pour des bases de données distribuées comme le
\emph{Bitcoin}, \emph{\href{https://datproject.org/}{Dat}} ou
\emph{\href{https://ipfs.io/}{IPFS}}, a le potentiel d'abaisser
d'avantage l'obstacle initial pour devenir responsable ou sous-traitant
des données sans en avoir conscience.

Pour permettre une adoption rapide des obligations en matière de
protection des données et, \emph{in fine} une augmentation générale de
l'hygiène autour données, une meilleure formation de tous les
responsables de traitement et de leurs sous-traitants est nécessaire, et
non pas seulement à ceux qui peuvent se permettre d'y consacrer les
ressources. Pour cette raison, nous estimons nécessaire la création
d'une base de données collaborative des connaissances, sur Internet, et
sous une licence \href{https://creativecommons.org/}{Creative Commons}
gratuite afin de s'assurer que sa disponibilité soit large et continue.

Jusqu'à présent, les conseils pratiques librement accessibles sont
souvent offerts par des parties prenantes avec des intérêts commerciaux.
Les prestataires de services en ligne, les cabinets d'avocats et les
organismes de formation orientent leurs conseils vers leurs propres
services. Les licences restrictives empêchent de partager librement les
conseils et les analyses. Les conseils nécessitent des mises à jour et
corrections qui ne sont pas toujours effectuées. Ce dernier point est
d'autant plus important que la conformité au RGPD est une cible
mouvante. Les nouvelles décisions ou les progrès techniques en matière
de protection de la vie privée\footnote{L'article 25 de la RGPD demande
  de prendre en compte notamment l'état de l'art lorsque lors de la
  conception et de la réalisation du traitement de données.} demandent
des mises à jour continues.

Comme la protection des données est un champ interdisciplinaire, la base
de connaissances devrait être construite par des professionnels du droit
et de l'informatique, et devrait répondre aux besoins des deux
communautés. La plateforme Stack Exchange fournit aux communautés une
solution logicielle pour construire une base de connaissances
collaborative sous la forme de questions-réponses. Cette plateforme est
bien connue des informaticiens via
\href{http://stackoverflow.com/}{stackoverflow.com} et a récemment
ouvert
\href{https://law.stackexchange.com}{law.stackexchange.com}\footnote{\href{https://law.stackexchange.com}{law.stackexchange.com}
  couvre en partie les questions de
  \href{https://law.stackexchange.com/questions/tagged/gdpr}{RGPD} and
  \href{https://law.stackexchange.com/questions/tagged/gdpr+data-protection}{protection
  des donnéesata protection}. Cependant, il nous semble que la
  protection des données personnelles nécessite une plateforme propre
  qui regroupe les autres disciplines comme l'informatique ou l'éthique.}.
La collaboration sur la plateforme est organisé de la façon suivante :

\begin{itemize}
\tightlist
\item
  Les questions, les réponses et leurs métadonnées sont publiées sur
  internet, sous une licence ouverte
  (\href{https://creativecommons.org/licenses/by-sa/3.0/}{CC BY-SA}) et
  sont téléchargeables dans un format interprétable par ordinateur
\item
  Tout le monde peut poser une question ou répondre
\item
  Les meilleurs réponses sont votées
\item
  L'utilisateur gagne de la réputation pour chaque vote reçu
\item
  L'utilisateur débloque des avantages au travers de la réputation
  reçue, comme la capacité de commenter ou voter
\item
  Les modérateurs sont élus parmis les utilisateurs et les utilisateurs
  les plus reconnus ont accès à des outils pour aider à la modération
\end{itemize}

Afin d'assurer la qualité des réponses, des références sont utilisées
lorsque des points sont discutables. Cette règle, également employée par
Wikipedia, est garantie par les modérateurs et les utilisateurs les plus
reconnus.

Les signataires soutiennent la création d'une telle base de données
collaborative de connaissances sur la protection des données sous forme
de questions fréquemment posées.

\textbf{Auteurs et signataires initiaux :}

\begin{itemize}
\tightlist
\item
  Robert Riemann, Bruxelles
\item
  Xavier Lavayssière, Paris
\item
  Franz Ritschel, Cologne
\end{itemize}

\textbf{Contact:}

Si vous voulez recevoir des mises à jours ou si vous avez une question,
vous pouvez nous contacter
\href{mailto:gdpr-faq@lesbricodeurs.fr}{\nolinkurl{gdpr-faq@lesbricodeurs.fr}}.
Si vous voulez signer la lettre, vous vous pouvez écrire à
\href{mailto:gdpr-faq-signer@lesbricodeurs.fr}{\nolinkurl{gdpr-faq-signer@lesbricodeurs.fr}}

\newpage

\textbf{Liste des destinataires:}

\begin{itemize}
\tightlist
\item
  \href{https://edps.europa.eu/data-protection/ipen-internet-privacy-engineering-network_en}{Internet
  Privacy Engineering Network}, une initiative du contrôleur européen de
  la protection des données
\item
  \href{https://stackoverflow.com/company}{Stack Overflow}, l'entreprise
  à l'origine de la base de connaissances
  \href{http://stackoverflow.com/}{stackoverflow.com}
\item
  \href{https://edri.org/}{European Digital Rights}, une association
  d'organisations de défense des droits civils et des droits de l'homme
  à travers l'europe
\item
  Les participants à l'édition 2018 du
  \href{http://privacyforum.eu/}{Annual Privacy Forum}
\item
  Le comité d'organisation de la conférence
  \href{http://www.cpdpconferences.org/}{Computers, Privacy and Data
  Protection}
\item
  Une liste de récipiendaires individuels
\end{itemize}

\end{letter}

\end{document}